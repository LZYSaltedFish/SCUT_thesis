\chapter{攻读博士/硕士学位期间取得的研究成果} %博士/硕士记得选其一
\pubfont % 论文撰写规范里,这章是5号宋体,\pubfont 设置字号为5号了。但其实很多论文用小四号也OK。

% 盲审版
% 一、已发表(包括已接受待发表)的论文,以及已投稿、或已成文打算投稿、或拟成文投稿的论文情况\underline{\textbf{(只填写与学位论文内容相关的部分):}}
% \begin{table}
% 	\centering{}%
% 	\pubfont 
% 	\begin{longtable}{|>{\centering}m{0.5cm}|m{4.2cm}|>{\centering}m{2.3cm}|>{\centering}m{1.8cm}|>{\centering}m{2cm}|>{\centering}m{1.5cm}|}
% 		\hline 
% 		\textbf{序号} & \textbf{发表或投稿刊物/会议名称} & \textbf{作者(仅注明第几作者)} & \textbf{发表年份} & \textbf{与学位论文哪一部分(章、节)相关} &\textbf{被索引收录情况}\tabularnewline
% 		\hline 
% 		1 & International Conference on Computational Linguistics (COLING), CCF-B类会议 & 共同第一作者 & 2024 & 第三章 & \tabularnewline
% 		\hline 
% 		2 & Annual Meeting of the Association for Computational Linguistics (ACL), CCF-A类会议 & 共同第一作者 & \begin{tabular}{c}
% 			2024 \\ (已投稿)
% 		\end{tabular} & 第三章 & \tabularnewline
% 		\hline 
% 	\end{longtable}
% \end{table}

% 注:1.请在“作者”一栏填写本人是第几作者,例:“第一作者”或“导师第一,本人第二”等;

%    2.若文章未发表或未被接受,请在“发表年份”一栏据实填写“已投稿”,“拟投稿”。

%    不够请另加页。
% \newline

% 二、与学位内容相关的其它成果(包括专利、著作、获奖项目等)

% 1. 发明专利:已受理一项发明专利,导师第一发明人,本人第二发明人,2024



% 非盲审版
一、已发表(包括已接受待发表)的论文,以及已投稿、或已成文打算投稿、或拟成文投稿的论文情况\underline{\textbf{(只填写与学位论文内容相关的部分):}}
\begin{table}
	\centering{}%
	\pubfont 
	\begin{longtable}{|>{\centering}m{0.4cm}|>{\centering}m{2.2cm}|>{\centering}m{3.9cm}|>{\centering}m{2.3cm}|>{\centering}m{1.8cm}|>{\centering}m{1.8cm}|>{\centering}m{1cm}|}
		\hline 
		\textbf{序号} & \textbf{作者(全体作者,按顺序排列)} & \textbf{题 \hspace{1.3em} 目} & \textbf{发表或投稿刊物名称、级别} & \textbf{发表的卷期、年月、页码} & \textbf{与学位论文哪一部分(章、节)相关} &\textbf{被索引收录情况}\tabularnewline
		\hline 
		1 & 李想,\textbf{李振宇}(共同一作),施晨,许勇,杜卿,谭明奎,黄俊 & AlphaFin: Benchmarking Financial Analysis with Retrieval-Augmented Stock-Chain Framework & International Conference on Computational Linguistics (COLING), CCF-B类会议 & \begin{tabular}{c} 2024,\\773–783 \end{tabular} & 第三章 & EI \tabularnewline
		\hline 
		2 & 李想,\textbf{李振宇}(共同一作),施晨,许勇,杜卿,谭明奎,黄俊 & FinAgent: Benchmarking Financial Analysis with Stock Agent Framework & Conference on Empirical Methods in Natural Language Processing (EMNLP),CCF-B类会议 & 2024 & 第三章 & 拟投稿 \tabularnewline
		\hline 
	\end{longtable}
\end{table}

注:1.请在“作者”一栏填写本人是第几作者,例:“第一作者”或“导师第一,本人第二”等;

   2.若文章未发表或未被接受,请在“发表年份”一栏据实填写“已投稿”,“拟投稿”。

   不够请另加页。
\newline

二、与学位内容相关的其它成果(包括专利、著作、获奖项目等)

1. 已受理发明专利:专利申请号:202410308335.3,专利名称:适用于多模态金融分析任务的端到端智能体金融模型方法



%注:这部分一言难尽,我努力了很久都没有把这个表做好。感觉学校给的这个表的模板非常反人类。看国外大学的博士论文,那种像参考文献著录信息那样一行一行的,比较美观。而这个框框很难放文字进去。

\normalsize % \normalsize可以将下文调回和正文一样的字号,这个随个人喜好。注释掉的话,致谢就就跟随《攻读博士/硕士学位期间取得的研究成果》的字号。