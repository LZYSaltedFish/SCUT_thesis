\chapter{致\texorpdfstring{\quad}{}谢}

在就读研究生的三年期间,我对人工智能领域有了更深刻的理解,除了丰富的知识和技术以外,我还收获了一段宝贵的人生经历。在这三年来的学习和生活中,我得到了许多来自家人、朋友、老师、同学的帮助和支持,在此论文即将完成之际,我希望对他们致以最诚挚的感谢。

首先我要衷心感谢我的指导老师,杜卿老师和谭明奎老师,他们在我的科研期间提供了许多指导和帮助,并对我的毕业论文尽心尽力指导修改。杜卿老师对学生认真负责,给予我们与各企业开展项目合作的机会,使我获得充分的学习与交流,同时在我企业联培期间,杜卿老师也定期召开工作会议,关心我的研究方向和研究进展,开拓我的科研思考方式。谭明奎老师科研态度严谨,从读论文、培养科研思维,到写论文、做展示、写Rebuttal,他都对学生悉心指导。他们都是我科研道路上的引路人,带领我迅速融入科研环境,使我得到极大的成长。

另外,非常感谢我的企业导师施晨博士,他的性格温和,做事认真细致,是我科研道路上的另一位引路人。在我实习期间,施晨老师在科学研究和实习工作上手把手指导我,对我的疑问一一耐心解答,还在生活中给了我很多鼓励和照顾,让我在学习和工作道路上有了前进的方向和动力。

我还要感谢与我合作进行课题研究的李想博士,他行动力强,对待科研充满热情。在我课题研究遇到困难时,他会帮助我一同寻找解决办法,平时时常与我讨论学术知识,为我的研究提供了许多思路和鼓励。

其次感谢其他帮助过我的同学,包括我的室友杨宗霖、杨泽杭、贺方舟,以及一同参与联培项目的曹庭锋、谢宇康、张宁、周纪咏、江宗源、林炜丰、汪嘉鹏、刘冰雁,他们给我的科研工作和学习生活提供了许多帮助,给我三年的研究生生活带来不少乐趣。另外,还要感谢实验室的各位同门同学,尤其是林坤阳和罗然同学,对我的毕业论文修改给出了不少建设性意见。

最后,我要特别感谢我的父母,在我读研的艰难时期,默默给予我鼓励与支持,陪我走过人生的灰暗。最后,再次对所有帮助和关心我的亲朋好友致以最诚挚的感谢和祝福。

% ~\\

% \begin{minipage}[t]{0.945\textwidth}%
% 	\begin{flushright}
% 		李振宇\\
% %		\today\\	% 自动时间
% 		2024年3月30日\\	%固定时间
% 		于华南理工大学
% 		\par\end{flushright}
% \end{minipage}
